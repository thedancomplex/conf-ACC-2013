 \section{Results \& Conclusion}\label{sec:con}
%


\begin{center}
	\begin{table}\label{table:resultsOpenLoop}
	\caption{Effects of changing the ratio of $J_L:J_a$ (Open Loop)}
	\begin{center}
  \begin{tabular}{  c | c | c | c }

Ratio					& $J_L$										& $\Delta dB$	& $\Delta$ TR Phase \\ 
($J_L:J_a$)		&($oz\cdot in\cdot sec^2$)& ($dB$)			& ($deg$)							\\	\hline
\hline
2.000  				&0.00460  								&9.542  			&180.0							\\	\hline
1.435  				&0.00330  								&7.729  			&180.0							\\	\hline
0.500  				&0.00115  								&3.522  			&180.0							\\

  \end{tabular}
  \end{center}
  \end{table}
\end{center}


\begin{center}
	\begin{table}\label{table:resultsClosedLoop}
	\caption{Effects of changing the ratio of $J_L:J_a$ when using RE and SMC (Closed Loop)}
	\begin{center}
  \begin{tabular}{ c | c | c | c | c | c }
						&								& 												& 								& $\Delta$ TR 		& $\Delta$ TR 			\\ 
Control			&Ratio					& $J_L$										& $\Delta dB$			& Mag 						& Phase 						\\ 
Method			&($J_L:J_a$)		&($oz\cdot in\cdot sec^2$)& ($dB$)					& ($dB$)					& ($deg$)						\\	\hline
\hline
RE 					&2.000  				&0.00460  								&0.500  					&0.400  					&4.600							\\	\hline
RE 					&1.435  				&0.00330  								&0.300  					&0.200  					&3.500							\\	\hline
RE 					&0.500  				&0.00115  								&0.500  					&0.100  					&1.600							\\	\hline
\hline
SMC 				&2.000  				&0.00460  								&0.000  					&24.652  					&130.840						\\	\hline
SMC 				&1.435  				&0.00330  								&0.000  					&10.200  					&23.770							\\	\hline
SMC 				&0.500  				&0.00115  								&0.000  					&0.012  					&1.815							\\	


  \end{tabular}
  \end{center}
  \end{table}
\end{center} 
The two different control designs for reducing the effect of TR in coupled systems, RE and SMC, both reduce the effect greatly. Table~\ref{table:resultsClosedLoop} shows the results from the control using RE and the control using SMC. 
These results show that RE reduces the $\Delta$ TR Mag over a greater range than SMC does. 
Both RE and SMC significantly reduce the $\Delta dB$.  
However SMC reduces the $\Delta dB$ to $0dB$ over the defined ratios of $J_L:J_a$ while RE continues to have the switching effect causing a $\Delta dB$ around $0.5dB$. 
It has also been shown that the SMC is effective for dealing with the load switching over a range of $J_L:J_a$ ratios.  
The performance and simplicity of RE makes it a good choice if you have access to the current $i_a$ and velocity $\dot{\theta}_a$ states. 
If do not have a current sensor but you do have access to the position $\theta_a$ and velocity $\dot{\theta}_a$ states then SMC is a valid choice.  
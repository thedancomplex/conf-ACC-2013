\section{Background}
A system with TR typically consists of three main items:
\begin{itemize}
\item An actuator, or motor, with a given inertia, Ja, and damping, Ba, associated with it.
\item An inertial load, JL, with a given damping, BL, associated with it.
\item A coupler between the actuator and the load with a spring constant, KC, and a damping, BC, associated with it.
\end{itemize}

It is assumed that the inertia of the coupler is included in the inertia of the actuator and the load, see Figure 4. The physical diagram of the system with TR can be found in Figure 3. In order to find the transfer function for this system, where Tin is the torque input and the actuator angle ?a is the output, the physical diagram is then converted into a mechanical network, see Figure 4, so the system�s dynamic equations can be written.

\begin{equation}
%T(t) = J_a\ddot{\theta_a}+\left( B_a+B_c \right) \dot{\theta_a}=K_c\theta_a-\left(K_c\theta_L+B_c\dot{\theta_L}\right)
T(t) = J_a\ddot{\theta_a}+(B_a+B_c ) \dot{\theta_a}=K_c\theta_a-(K_c\theta_L+B_c\dot{\theta_L})
\end{equation}

and



\begin{equation}\label{eq:thetaLthetaa}
\frac{\theta_L}{\theta_a} = \frac{K_c+sB_c}{s^2+J_L+sB_L+K_c+sB_c}
\end{equation}

is the relationship between $\theta_L$ and $\theta_a$.  \ref{eq:thetaLthetaa} shows the relationship between $\theta_L$ and $\theta_a$ as the ratio of $\frac{\theta_L}{\theta_a}$. Note that $\theta_a$ contains up to and including first order terms and ?L includes up to and including second order terms.


